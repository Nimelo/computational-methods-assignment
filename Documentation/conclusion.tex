%----------------------------------------------------------------------------------------
%	CONCLUSIONS
% • What the author thinks the results mean based on the observations.
% • Should relate directly back to the problem/question stated in the introduction.
% • DO NOT INTRODUCE SURPRISES HERE
%----------------------------------------------------------------------------------------

\chapter{Conclusion}
	During examination of results some of properties and major facts were discovered. For first order schemes - explicit and implicit upwind dissipation error was noticed. Similarly for second-order schemes dispersion error was observed. For which magnitude of those errors vary with CFL numbers. It is good practice to keep $\Delta x$ and $\Delta t$ values as small as possible, but it doesn't always give better solution. In some cases use of different scheme can change the overall error without changing discretization parameters. For example $\Delta t$ can remain the same (relatively small) for getting more accurate solution in terms of time level. The most important thing is to provide stability for the scheme, that's why in first order all the schemes should be analyzed for example with von Neumann stability analysis. The solutions of schemes in terms of unstable coefficients are generally meaningless but it does not mean that such approach is not used anywhere. It is due to unpredictable behavior of unstable schemes. Unconditionally stable schemes are good because it give wider spectrum of values to choose from the optimal CFL number. Although in terms of implicit upwind scheme discovery were made, such for positive numbers solution is not acceptable. The more optimal CFL number we choose the smaller accumulated error we get for longer time steps. Analysis of results showed also that all of the schemes are convergent with increase of grid points. Magnitude of error also vary not only on scheme and discretization parameters, it also vary on function that is discretized.
	
	Summarizing parametrization of discretization process is quite complex process, which requires solid knowledge not only about solving such problem, but also knowledge about domain of a problem. Fit of a scheme is sometimes hard, because some properties or errors that scheme carries out can unacceptable but generally gives small error and better result. Schemes used in this report work on homogeneous grid, which means that density of points is the same for whole function. Changing density of points in grid can improve results around discontinuities, but makes whole process even more complex. 