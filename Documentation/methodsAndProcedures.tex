%----------------------------------------------------------------------------------------
%	METHODS AND PROCEDURES
% • The experimental design, methods of gathering data.
% • Subsections.
%----------------------------------------------------------------------------------------
\section{General approach}
	Different approaches have been proposed to describe finite difference method in many books. In case of this report I will base on explanation in chapter 3.5 in Computational Techniques for Fluid Dynamics\cite{bib:fletcher} written by C. A. J. Fletcher. The basis for the finite difference method is the construction of a discrete grid, the replacement of the continuous derivatives in the governing partial equations with equivalent finite difference expressions. Mentioned transformation is presented in Figure \ref{fig:schematicOfFDM}\cite{bib:fletcher}[p. 64--65].
	
	\begin{figure}
		\centering
		\begin{tikzpicture}%[node distance = 3cm, auto]
			% Place nodes
			\node [block] (setUpGrid) {Set up grid};
			\node [block, right of=setUpGrid, align=center] (initDependentVars) 
			{
				Initialise \\ 
				dependent \\ 
				vairables
			};
			\node [block, right =1cm of initDependentVars, align=center] (construct)
			{
				Construct finite difference \\
				analogue of P.D.E and B.C.s
			};			
			\node [block, below of=construct, align=center] (forEach)
			{
					 For each interior grid point (j, n) \\
					 evaluate algorithm to give $T_j^{n+1}$
			};			
			\node [block, left=1cm of forEach, below of=initDependentVars] (timeStep) 
			{
					$t_{n+1} = t_n + \Delta t$
			};
			\node [block, below of=timeStep, align=center] (adjust) 
			{
					Adjust (if necessary) \\
					boundary values \\
					$T_1^{n+1}$ and $T_j^{n+1}$
			};
			\node [decision, left=1cm of adjust] (final) {Final time reached};
			\node [block, below of=final] (solution) {Solution};
			% Draw edges
			\path [line] (setUpGrid) -- (initDependentVars);
			\path [line] (initDependentVars) -- (construct);
			\path [line] (construct) -- (forEach);
			\path [line] (timeStep) -- (forEach);
			\path [line] (forEach) |- (adjust);
			\path [line] (adjust) -- (final);
			\path [line] (final) |- node {no} (timeStep);
			\path [line] (final) -- node {yes} (solution);
		\end{tikzpicture}
		\caption{Schemataic of the finite difference solution process.}
		\label{fig:schematicOfFDM}
	\end{figure}