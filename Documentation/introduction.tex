%----------------------------------------------------------------------------------------
%	INTRODUCTION
% • Statement of the problem investigated.
% • Background information.
% • Concise introduction to theory or concepts used.
% • Subsections.
%----------------------------------------------------------------------------------------
\section{Introduction} \label{sec:Intoduction}
	The numerical methods are fundamental to solve many problems from a real-world. In case of this assignment all work is focused around advection equation which is a physic formula that describes transport of a substance. This is not only branch of science that uses numerical methods in order to solve problems. It is true that physics, engineering and earth sciences are extremely popular branches of science that use numerical approximations for many practical problems. For example in meteorology and physical oceanography, advection often refers to the horizontal transport of some property of the atmosphere or ocean, such as hear, humidity or salinity, and convection generally refers to vertical transport (vertical advection)\footnote{https://en.wikipedia.org/wiki/Advection}. In case of this assignment we talk about about horizontal transport of wave in time. Apart of the problem there are many numerous schemes designed to ensure the best results. There are many existing powerful and flexible tools, but the user of them should be highly aware of the consequences of using them. A key aspect of using numerical schemes is to balance between good accuracy and stability of the scheme.
	
	Stability of the scheme has a pivotal role in ensuring that the solution is meaningful. It also means that we can trust our results, because there is no way to compare our numerical results - there is no analytical solution. The second crucial element for good calculation is examination of grid convergence. If used schema is not convergent, it cannot be used in a practical way. Every schema distorts our solution by introducing vary effects, such as: dissipation or dispersion. Besides of those effects each of the schemes also introduces truncation error. Apart of all factors related to the schemes, there is also a round-off error related to the precision of a computer which may significantly affect our results.
	
	\subsection{Objective} \label{sec:Objective}
		The main objective of the assignment is to implement, examine and discuss about four numerical schemes (implicit and explicit upwind, Lax-Wendroff and Richtmyer multi-step) used to solve simple advection equation which is described as follows:
		
		\begin{equation}
			\label{for:advection}
			\frac{\partial f}{\partial t} +  u\frac{\partial f}{\partial x} = 0
		\end{equation}
		
		in a domain $x \in [-50,50]$ with $u = 1.75$ and two initial/boundary conditions:
		
		\begin{subequations}
			\begin{align}
				\begin{split}
					\label{for:firstSet}
					f(x, 0) &= \frac{1}{2} (sign(x) + 1) \\
					f(-50, t) &= 0 \\
					f(50, t) &= 1
				\end{split} \\
				\begin{split}
					\label{for:secondSet}
					f(x, 0) &= \frac{1}{2} exp(-x^2) \\
					f(-50, t) &= 0 \\
					f(50, t) &= 0
				\end{split}
			\end{align}
		\end{subequations}
	
		The analytical solutions for these initial conditions are given respectively:
		
		\begin{subequations}
			\begin{align}
			\begin{split}
				f(x, t) = \frac{1}{2} (sign(x - 1.75t) + 1)
			\end{split} \\
			\begin{split}
				f(x, t) = \frac{1}{2} exp(-(x - 1.75t)^2).
			\end{split}
			\end{align}
		\end{subequations}
	
		Mentioned before analytical solutions and known are will be used in order to the calculate an error. Further analyses will compare errors of each scheme to determine the best numerical solution between them.