\section{Formulation and stability analysis of explicit upwind scheme}
	Formula that gives first-order accurate method for our problem is described by Klaus A. Hoffman and Steve T. Chiang in \cite{bib:hoffman}[p. 191 - 192] in section 6.5 titled Applications to a Linear Problem, which looks as follows:
		\begin{align}
			\label{app:for:explicitUpwind}
			\begin{split}
				\frac{u_j^{n+1} - u_j^n}{\Delta t} + a\frac{u_j^n - u_{j-1}^n}{\Delta x} = \mathcal{O}(\Delta t, \Delta x)
			\end{split}
		\end{align}
	where $j$ is related to discretized point, $a$ is acceleration of advection function and $n$ is current time step. This scheme uses Forward-Time Backward-Space (FTBS) discretization.
	The only unknown variable in (\ref{app:for:explicitUpwind}) is $f_i^{n+1}$, because it is explicit scheme, for which we can compute solution without solving set of equation. Solution for unknown presents:
	
		\begin{align}
			\label{app:for:explicitUpwindSolution}
			\begin{split}
				u_j^{n+1} = u_j^n - C(u_j^n - u_{j-1}^n) + \mathcal{O}(\Delta t^2, \Delta x \Delta t)
			\end{split}			
		\end{align}
	where $C=\frac{u\Delta t}{\Delta x}$ is Courant-Friedrichs-Lewy (CFL) number.	
	From equation (\ref{app:for:explicitUpwindSolution}) von Neumann Analysis will be carried out. Starting with following equation:
	\begin{equation}
		\label{app:for:explicit}
		u_j^{(n+1)} = u_j^{(n)} - C\Big(u_j^{(n)} - u_{j-1}^{(n)}\Big)
	\end{equation}	
	Key to the von Neumann's analysis is assuming that the solution to the difference equation is of the form:
	\begin{equation}
		\label{app:for:vnkey}
		u_j^{(n)} = \sigma^ne^{ikx_j}
	\end{equation}	
	Recall that $e^{ikx_j} = cos(kx_j) + isin(kx_j)$. Substituting (\ref{app:for:vnkey}) into (\ref{app:for:explicit}) results in:
	
	\begin{align}
		\begin{split}
			\sigma ^{n+1}e^{ikx_j} = \sigma^{n}{e^{ikx_j}} - C(\sigma ^{n}e^{ikx_j} - \sigma ^{n}e^{ikx_{j-1}}) \\
		\end{split}
	\end{align}
	Noting that	
	\begin{align}
		\begin{split}
			x_{j+1} &= x_j + \Delta x \\
			x_{j-1} &= x_j - \Delta x
		\end{split}
	\end{align}	
	and dividing through by $u_j^{(n)} = \sigma^ne^{ikx_j}$ we obtain
	\begin{align}
		\begin{split}
			\sigma &= 1 - C\Big(1 - \frac{e^{ikx_{j-1}}}{e^{ikx_j}}\Big) \\
			\sigma &= 1 - C\Big(1 - \frac{e^{ik (x_j - \Delta x)}}{e^{ikx_j}}\Big) \\
			\sigma &= 1 - C\Big(1 - e^{-ik\Delta x}\Big) \\
		\end{split}
	\end{align}
	The method is stable if the amplification factor is less or equal to 1. The magnitude of the amplification factor is:
	\begin{align}
		\begin{split}
			\label{app:for:sigmaExplicit}
			|\sigma|^2 &= \Bigg( 1 - C\Big(1 - e^{-ik\Delta x}\Big)\Bigg)^2 \\
			&= \Bigg( 1 - C\Big(1 -\cos(k\Delta x) + i\sin(k\Delta x)\Big)\Bigg)^2 \\
			&= \Bigg(1 - C\Big(1 -\cos(k\Delta x)\Big) - iC\sin(k\Delta x)\Bigg)^2 \\	
			&= 1 - 2C\Big(1 - \cos(k\Delta x)\Big) + C^2\Big(1 -\cos(k\Delta x)\Big)^2 + C^2\sin^2(k\Delta x)
		\end{split}
	\end{align} 	
	but	
	\begin{align}
		\begin{split}
			\label{app:for:dependencyExplicit}
			C^2\Big(1 -\cos(k\Delta x)\Big)^2 + C^2\sin^2(k\Delta x) &=C^2\Bigg(1 - 2\cos(k\Delta x) + \cos^2(k\Delta x)\Bigg) + C^2\sin^2(k\Delta x) \\
			&= C^2(1-2\cos(k\Delta x)) + C^2 \\
			&= 2C^2(1-\cos(k\Delta x)))
		\end{split}
	\end{align} 	
	Using dependency (\ref{app:for:dependencyExplicit}) for (\ref{app:for:sigmaExplicit}) we obtain:	
	\begin{align}
		\begin{split}
			|\sigma|^2 &= 1 - 2C\Big(1-\cos(k\Delta x)\Big) + 2C^2\Big(1 - \cos(k\Delta x)\Big) \\
			&= 1-2C\Big(1-\cos(k\Delta x)\Big)(1-C)
		\end{split}
	\end{align} 	
	The method is stable if $|\sigma|^2 \leq 1$, our equation is presented in a following way:	
	\begin{align}
		\begin{split}
			1-2C\Big(1-\cos(k\Delta x)\Big)(1-C) &\leq 1 \\
			-2C\Big(1-\cos(k\Delta x)\Big)(1-C) &\leq 0 \\
			2C\Big(1-\cos(k\Delta x)\Big)(1-C) &\leq0 \\
		\end{split}
	\end{align} 	
	Knowing that $ \forall k,  1 \geq \cos(k\Delta x) \geq -1$, we can obtain final result:	
	\begin{align}
		\begin{split}
			\begin{cases}
				(1-C) \leq 0 \implies C \leq 1 \\
				C \geq 0
			\end{cases}
		\end{split}
	\end{align} 	
	Summarizing this scheme is conditionally stable with following condition $C \in \left(0, 1\right)$.
	
	
	