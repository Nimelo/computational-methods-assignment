\subsection{Richtmyer multi-step}
	Richtmyer multi-step scheme in terms of assignment is stable for following condition: $C \in (0,2)$. In Figures \ref{fig:richtmyerSignCFL} and \ref{fig:richtmyerExpCFL} results of numerical solution are shown, respectively for initial solution sets (\ref{for:firstSet}) and (\ref{for:secondSet}). Figure \ref{fig:richtmyerfCFLNorms} presents distribution of an error among different CFL numbers.
	\begin{figure}[!htbp]
		\centering	
		\begin{tikzpicture}[scale=0.5]	
		\pgfplotsset{width=\textwidth}
			\begin{axis}[
				xlabel = {$x$},
				ylabel = {$y$},
				%ymin = -1, ymax = 1,
				xmin = 2, xmax = 32,
				minor y tick num = 1,
				ymajorgrids=true,
				xmajorgrids=true,
				grid style=dashed,
				legend pos=north west,
				ticklabel style={
					/pgf/number format/fixed,
					/pgf/number format/precision=2
				},
				]
			
				\addcustomplot{../Code/results/impactRichtmyer-Multi-Step/richtmyer-ms-100-sign.conf-chart.csv}{red}{4}{Analytical}
				\addcustomplot{../Code/results/impactRichtmyer-Multi-Step/richtmyer-ms-100-sign.conf-chart.csv}{green}{3}{$CFL=1.96$}
				\addcustomplot{../Code/results/impactRichtmyer-Multi-Step/richtmyer-ms-100-sign-CFL-1.9.conf-chart.csv}{purple}{3}{$CFL=1.9$}
				\addcustomplot{../Code/results/impactRichtmyer-Multi-Step/richtmyer-ms-100-sign-CFL-0.8.conf-chart.csv}{blue}{3}{$CFL=0.8$}
			\end{axis}
		\end{tikzpicture}
		\caption{Richtmyer multi-step scheme solutions for the initial set (\ref{for:firstSet}) and various CFL numbers.}
		\label{fig:richtmyerSignCFL}
	\end{figure}
	Similarly to the Lax-Wendroff scheme Richtmyer multi-step method introduces dispersion errors, because mentioned scheme is second-order accurate in time and space. Alike to Lax-Wendroff scheme dispersion for this scheme also occurs for small CFL numbers. Solutions for (\ref{for:firstSet}) give smooth and continuous curves for smaller CFLs. Continuousness disappears with increase of CFL number. Shape and behavior of solutions for (\ref{for:secondSet}) is unexpected. Dispersion occurs near discontinuous fragment of curve, similarly like for Lax-Wendroff scheme. The second anomaly is saw-shaped top fragment of the curve. Both of mentioned anomalies have lower impact on solution for bigger CFLs.
	\begin{figure}[!hbtp]
		\centering	
		\begin{tikzpicture}[scale=0.5]
		
			\pgfplotsset{width=\textwidth}
			\begin{axis}[
				xlabel = {$x$},
				ylabel = {$y$},
				%ymin = -1, ymax = 1,
				xmin = 2, xmax = 32,
				minor y tick num = 1,
				ymajorgrids=true,
				xmajorgrids=true,
				grid style=dashed,
				ticklabel style={
					/pgf/number format/fixed,
					/pgf/number format/precision=2
				},
				]
				
				\addcustomplot{../Code/results/impactRichtmyer-Multi-Step/richtmyer-ms-100-exp.conf-chart.csv}{red}{4}{Analytical}
				\addcustomplot{../Code/results/impactRichtmyer-Multi-Step/richtmyer-ms-100-exp.conf-chart.csv}{green}{3}{$CFL=1.995$}
				\addcustomplot{../Code/results/impactRichtmyer-Multi-Step/richtmyer-ms-100-exp-CFL-1.9.conf-chart.csv}{purple}{3}{$CFL=1.9$}
				\addcustomplot{../Code/results/impactRichtmyer-Multi-Step/richtmyer-ms-100-exp-CFL-0.8.conf-chart.csv}{blue}{3}{$CFL=0.8$}
			\end{axis}
		\end{tikzpicture}
		\caption{Richtmyer multi-step scheme solutions for the initial set (\ref{for:secondSet}) and various CFL numbers.}
		\label{fig:richtmyerExpCFL}
	\end{figure} 
	\begin{figure}[!htbp]
		\begin{subfigure}[b]{0.5\textwidth}
			\begin{tikzpicture}
			%\pgfplotsset{width=\textwidth}
				\begin{axis}[
					ybar,
					%symbolic x coords={5,10,15,20},
					xticklabels={$t=5$,$t=10$},
					xtick=data,
					enlarge x limits={abs=2cm},
					ymajorgrids=true,
					xmajorgrids=true,
					grid style=dashed,
					%legend pos=north west
					legend style={at={(0.5,-0.1)},anchor=north,legend cell align=left}
					%nodes near coords,
					%every node near coord/.append style={rotate=90, anchor=west},
					]
					\addcustomybarplot{../Code/results/impactRichtmyer-Multi-Step/richtmyer-ms-100-exp.conf-norms.csv}{2}{green}{$CFL=1.995$};
					\addcustomybarplot{../Code/results/impactRichtmyer-Multi-Step/richtmyer-ms-100-exp-CFL-1.9.conf-norms.csv}{2}{blue}{$CFL=1.9$};
					\addcustomybarplot{../Code/results/impactRichtmyer-Multi-Step/richtmyer-ms-100-exp-CFL-0.8.conf-norms.csv}{2}{purple}{$CFL=0.8$};
				
				\end{axis}
			\end{tikzpicture}
			\caption{Initial set (\ref{for:secondSet}).}
		\end{subfigure}
		\begin{subfigure}[b]{0.5\textwidth}
			\begin{tikzpicture}
				\begin{axis}[
					ybar,
					%symbolic x coords={5,10,15,20},
					xticklabels={$t=5$,$t=10$},
					xtick=data,
					enlarge x limits={abs=2cm},
					ymajorgrids=true,
					xmajorgrids=true,
					grid style=dashed,
					%legend pos=north west
					legend style={at={(0.5,-0.1)},anchor=north,legend cell align=left},
					%nodes near coords,
					%every node near coord/.append style={rotate=90, anchor=west},
					]
					\addcustomybarplot{../Code/results/impactRichtmyer-Multi-Step/richtmyer-ms-100-sign.conf-norms.csv}{2}{green}{$CFL=1.96$};
					\addcustomybarplot{../Code/results/impactRichtmyer-Multi-Step/richtmyer-ms-100-sign-CFL-1.9.conf-norms.csv}{2}{blue}{$CFL=1.9$};
					\addcustomybarplot{../Code/results/impactRichtmyer-Multi-Step/richtmyer-ms-100-sign-CFL-0.8.conf-norms.csv}{2}{purple}{$CFL=0.8$};
				
				\end{axis}
			\end{tikzpicture}
			\caption{Initial set (\ref{for:firstSet}).}
		\end{subfigure}
		\caption{Distribution of the error (norm 2) among different CFL numbers for Richtmyer multi-step scheme.}
		\label{fig:richtmyerfCFLNorms}
	\end{figure}
	Alike in all previous mentioned explicit schemes error grows with the decrease of CFL number. Worth of noticing is a difference between two highest CFL numbers for (\ref{for:secondSet}). Difference between this CFL numbers is equal to $0.095$ which translates into $0.5 * 10^2$ increase of error. Surprisingly decrease of CFL number from $1.9$ to $0.8$ worsen solution by $1 * 10^-2$, which is not as intuitive as it is. Initial set (\ref{for:firstSet}) has worse results in terms of calculated error comparing to the other one.
	\clearpage