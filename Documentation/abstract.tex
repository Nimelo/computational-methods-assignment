	%----------------------------------------------------------------------------------------
	%	ABSTRACT
	% • Self-explanatory without reference to the paper.
	% • Max 2 paragraphs, NO ABBREVIATIONS HERE.
	% • Concisely indicate the experiment, objectives, importance.
	% • Newly observed facts and the conclusions of the experiment.
	% • Only the most significant results should be given.
	%----------------------------------------------------------------------------------------
	\pagenumbering{arabic}
	\begin{abstract}
		Imitation of the operation of a real-world processes or system over time is very often used by almost all engineers. Those imitation is called simulation and is used because some processes cannot be engaged because some of them are not accessible or it may be dangerous or unacceptable to engage, sometimes this processes doesn't exist at all. Numerical methods can be really useful and powerful tools for solving differential equations specified by those processes. This report summarizes and explains work about four common numerical schemes: explicit upwind, implicit upwind, Lax-Wendroff and Richtmyer multi-step to solve a simple linear advection equation. For each of schemes calculation were made for different delta time, time levels and grid sizes. All of schemes and other necessary code were written in pure C++ language.
		
		Linear advection equation is an example of partial differential equation for time and space. Applying schemes for different set of parameters showed that every numerical solution have to struggle with errors and undesirable effects, such as: dissipation, dispersion and instability. Accumulation of the error and overall loss of accuracy were due to increase of time steps. All of mentioned schemes are convergent with the increase of the grid size.
		
		%FIX THIS
		
	\end{abstract}