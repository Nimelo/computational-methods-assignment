	%----------------------------------------------------------------------------------------
	%	ABSTRACT
	% • Self-explanatory without reference to the paper.
	% • Max 2 paragraphs, NO ABBREVIATIONS HERE.
	% • Concisely indicate the experiment, objectives, importance.
	% • Newly observed facts and the conclusions of the experiment.
	% • Only the most significant results should be given.
	%----------------------------------------------------------------------------------------
	\pagenumbering{arabic}
	\begin{abstract}
		Imitation of the operation of real-world processes or system over time is very often used by almost all engineers. Those imitations are called simulations and are used because some processes cannot be engaged. Some of them are not accessible or may be dangerous and unacceptable to engage, hence sometimes these processes don't exist at all. Numerical methods are very useful and powerful tools for solving differential equations specified by these processes. This report summarizes and explains four common numerical schemes: explicit upwind, implicit upwind, Lax-Wendroff and Richtmyer multi-step to solve a simple linear advection equation. For each of these schemes calculation were made for different step sizes, time levels and grid sizes. Solutions to these schemes and programming were written in pure C++ language.
		
		Linear advection is an example of partial differential equation for time and space. Applying schemes for different set of parameters showed that every numerical solution have to struggle with errors and undesirable effects such as: dissipation, dispersion and instability. Accumulation of the error and overall loss of accuracy were due to increase in step sizes. The accuracy of the schemes and how the numerical solutions vary with the CFL number (Courant-Friedrichs-Lewy). Also examining how these schemes converge as the grid size varies.
		
		%IS IT TRUE?
		In the last section, comparison between schemes were made. All of the schemes have unique properties that were compared to the expected properties based on the knowledge in the literature. Schema with the biggest tolerance of CFL number is implicit upwind because it is stable for a wide range of values. Other schemes have more restricted criteria of stability, which makes them very vulnerable to losing control due to an increase in error. Richtmyer multi-step and Lax-Wendroff schemes are comparable and have the highest accuracy due to applying appropriate corrector method with each time step. The best results were acquired by explicit upwind scheme for signum initialization function and Lax-Wendroff scheme for exponential function.
	\end{abstract}